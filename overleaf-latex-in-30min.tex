% this defines the type of document, i.e. class. Inside [] are font size, page size & others settings
\documentclass[12pt, letterpaper, twoside]{article}

% encoding of document
\usepackage[utf8]{inputenc}

% for adding image
\usepackage{graphicx}
\graphicspath{{./}}

% preamble above

\begin{document}

\title{First \LaTeX{} document}
\author{SZ \thanks{For my love ones}}
\today
% generate title as in preamble
\maketitle

\tableofcontents

\begin{abstract}
This is a full demonstration of all the example contents in Latex in 30 minutes online from Overleaf.
\end{abstract}

\section{Bold, italics, underline}

% bold, italics(\textit{} or \emph{}) & underline
\textbf{Latex} is \underline{widly} used and recognized by its \textbf{\textit{logo}} as in figure \ref{fig:logo}. 

\section{Add image}

\begin{figure}[h]
    \centering
    \includegraphics[width=0.25\textwidth]{latex}
    \caption{latex logo}
    \label{fig:logo}
\end{figure}

\section{Unordered list}
% unordered list
\begin{itemize}
  \item The individual entries are indicated with a black dot, a so-called bullet.
  \item The text in the entries may be of any length.
\end{itemize}

\section{Ordered list}
% ordered list
\begin{enumerate}
  \item This is the first entry in our list
  \item The list numbers increase with each entry we add
\end{enumerate}

\section{Math equations}
% add math equation inline (use one of these delimiters: \( ... \), $ ... $ or \begin{math} ... \end{math})
Albert Einstein discovered the equation $E=mc^2$ in 1905.\\
Michelangelo was born on March 6\textsuperscript{th}, 1475.


% add math equation, numbered and unnumbered, in display mode (use one of these delimiters: \[ ... \], \begin{displaymath} ... \end{displaymath} or \begin{equation} ... \end{equation})
The mass-energy equivalence is described by the famous equation
\[ E=mc^2 \]
discovered in 1905 by Albert Einstein. 
In natural units ($c = 1$), the formula expresses the identity
\begin{equation}
E=m
\end{equation}

Subscripts in math mode are written as $a_b$ and superscripts are written as $a^b$. These can be combined an nested to write expressions such as

\[ T^{i_1 i_2 \dots i_p}_{j_1 j_2 \dots j_q} = T(x^{i_1},\dots,x^{i_p},e_{j_1},\dots,e_{j_q}) \]
 
We write integrals using $\int$ and fractions using $\frac{a}{b}$. Limits are placed on integrals using superscripts and subscripts:

\[ \int_0^1 \frac{dx}{e^x} =  \frac{e-1}{e} \]

Lower case Greek letters are written as $\omega$ $\delta$ etc. while upper case Greek letters are written as $\Omega$ $\Delta$.

Mathematical operators are prefixed with a backslash as $\sin(\beta)$, $\cos(\alpha)$, $\log(x)$ etc.

\section{Paragraph}
% paragraphs and newlines
% note how to add quote mark below
Hit the `Enter' key twice (to insert a double blank line) to start a new paragraph.


To start a new line without actually starting a new paragraph insert a \textit{break line} point, this can be done by \textbackslash{}\textbackslash{} (a double backslash as in the example) or the \textbf{\textbackslash{}newline} command. 


The recommended method to `simulate' paragraphs with larger spacing between them is to keep using double blank lines to create new paragraphs without any \textbackslash{}\textbackslash{}, and then add \textbackslash{}usepackage\{parskip\} to the preamble. 


\addcontentsline{toc}{section}{Unnumbered Section}
\section*{Unnumbered Section}
Just an example.


\section{Tables}
Tables \ref{table:data1} \ref{table:data2} and \ref{table:data3} are examples of referenced elements.

\subsection{No boarder}
\begin{table}[h!]
\begin{center}
\begin{tabular}{ c c c }
 cell1 & cell2 & cell3 \\ 
 cell4 & cell5 & cell6 \\  
 cell7 & cell8 & cell9    
\end{tabular}
\end{center}
\caption{Table without boarder}
\label{table:data1}
\end{table}

\subsection{With boarder 1}
\begin{table}[h!]
\begin{center}
\begin{tabular}{ |c|c|c| } 
 \hline
 cell1 & cell2 & cell3 \\ 
 cell4 & cell5 & cell6 \\ 
 cell7 & cell8 & cell9 \\ 
 \hline
\end{tabular}
\end{center}
\caption{Table without boarder}
\label{table:data2}
\end{table}

\subsection{With boarder 2}
\begin{table}[h!]
\begin{center}
 \begin{tabular}{||c c c c||} 
 \hline
 Col1 & Col2 & Col2 & Col3 \\ [0.5ex] 
 \hline\hline
 1 & 6 & 87837 & 787 \\ 
 \hline
 2 & 7 & 78 & 5415 \\
 \hline
 3 & 545 & 778 & 7507 \\
 \hline
 4 & 545 & 18744 & 7560 \\
 \hline
 5 & 88 & 788 & 6344 \\ [1ex] 
 \hline
\end{tabular}
\end{center}
\caption{Table without boarder}
\label{table:data3}
\end{table}

\section{Dashes and hyphens}
Hyphen: daughter-in-law, X-rated\\
En dash: pages 13--67\\
Em dash: yes---or no? \\
Minus sign: $0$, $1$ and $-1$

\section{Ellipsis}
Not like this ... but like this:\\
New York, Tokyo, Budapest, \ldots

\end{document}
